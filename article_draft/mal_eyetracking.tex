\documentclass{article}


\usepackage{arxiv}

\usepackage[utf8]{inputenc} % allow utf-8 input
\usepackage[T1]{fontenc}    % use 8-bit T1 fonts
\usepackage{hyperref}       % hyperlinks
\usepackage{url}            % simple URL typesetting
\usepackage{booktabs}       % professional-quality tables
\usepackage{amsfonts}       % blackboard math symbols
\usepackage{nicefrac}       % compact symbols for 1/2, etc.
\usepackage{microtype}      % microtypography
\usepackage{lipsum}

\title{Lexical access within a dense coronal contrast system}


\author{
  Indranil Dutta\thanks{The names of the authors appear alphabetically by last name. The study was conceptualized by all the authors together. The stimuli for the eye-tracking study was prepared by the second author. The eye-tracking experiment was conducted by the third author. The first, second, and fourth authors wrote different sections of the paper relevant to their contributions.} \\
  Speech and Language Processing Laboratory \\
  Department of Computational Linguistics\\
  The EFL University\\
  Hyderabad, TS 500007 \\
  \texttt{indranil@efluniversity.ac.in} \\
  %% examples of more authors
   \And
   Meghavarshini Krishnaswamy\\
  Speech and Language Processing Laboratory \\
  Department of Computational Linguistics\\
  The EFL University\\
  Hyderabad, TS 500007\\
  \texttt{meghavarshini@gmail.com } \\
  \And
  Ramesh Kumar Mishra \\
  Action Control and Cognition Lab \\
  Center for Neural and Cognitive Sciences (CNCS) \\
  University of Hyderabad\\
  Hyderabad, TS 500046\\
  \texttt{rkmishra@uohyd.ac.in} \\
  \And
  Seema Gorur Prasad \\
  Action Control and Cognition Lab \\
  Center for Neural and Cognitive Sciences (CNCS) \\
  University of Hyderabad\\
  Hyderabad, TS 500046\\
  \texttt{gp.seema@gmail.com} \\
  %% \AND
  %% Coauthor \\
  %% Affiliation \\
  %% Address \\
  %% \texttt{email} \\
  %% \And
  %% Coauthor \\
  %% Affiliation \\
  %% Address \\
  %% \texttt{email} \\
  %% \And
  %% Coauthor \\
  %% Affiliation \\
  %% Address \\
  %% \texttt{email} \\
}

\begin{document}
\maketitle

\begin{abstract}
This is where the abstract will come.
\end{abstract}


% keywords can be removed
\keywords{Coronal contrasts, lexical access, visual world, Malayalam}


\section{Introduction}
This is where the introduction will come. Assigned to ID.


\section{Materials and methods}
\label{sec:MandM}

This is where the materials and methods will come.

\subsection{Materials}
This is where the materials will come. Assigned to MK.

\subsection{Methods}
This is where the methods will come. Assigned to SGP.

\section{Results}
\label{sec:results}
This where the results will come. Assigned to ID, MK, and SGP. Add and cite references from malret.bib. For example: \cite{adank04,tabain1999,pandey2006}. If you are using SublimeText as your text editor along with LatexTools then enable `traditional' as the bibliography in the user settings file and it will automatically complete citations from the bib file.

\section{Discussion}
\label{sec:disc}
This is where the discussion section will come. Assigned to ID.

% The documentation for \verb+natbib+ may be found at
% \begin{center}
%   \url{http://mirrors.ctan.org/macros/latex/contrib/natbib/natnotes.pdf}
% \end{center}
% Of note is the command \verb+\citet+, which produces citations
% appropriate for use in inline text.  For example,
% \begin{verbatim}
%    \citet{hasselmo} investigated\dots{}
% \end{verbatim}
% produces
% \begin{quote}
%   Hasselmo, et al.\ (1995) investigated\dots
% \end{quote}

% \begin{center}
%   \url{https://www.ctan.org/pkg/booktabs}
% \end{center}


% \subsection{Figures}
% \lipsum[10] 
% See Figure \ref{fig:fig1}. Here is how you add footnotes. \footnote{Sample of the first footnote.}
% \lipsum[11] 

% \begin{figure}
%   \centering
%   \fbox{\rule[-.5cm]{4cm}{4cm} \rule[-.5cm]{4cm}{0cm}}
%   \caption{Sample figure caption.}
%   \label{fig:fig1}
% \end{figure}

% \subsection{Tables}
% \lipsum[12]
% See awesome Table~\ref{tab:table}.

% \begin{table}
%  \caption{Sample table title}
%   \centering
%   \begin{tabular}{lll}
%     \toprule
%     \multicolumn{2}{c}{Part}                   \\
%     \cmidrule(r){1-2}
%     Name     & Description     & Size ($\mu$m) \\
%     \midrule
%     Dendrite & Input terminal  & $\sim$100     \\
%     Axon     & Output terminal & $\sim$10      \\
%     Soma     & Cell body       & up to $10^6$  \\
%     \bottomrule
%   \end{tabular}
%   \label{tab:table}
% \end{table}



\bibliographystyle{unsrt}  
%\bibliography{references}  %%% Remove comment to use the external .bib file (using bibtex).
%%% and comment out the ``thebibliography'' section.


%%% Comment out this section when you \bibliography{mal_ret.bib} is enabled.
\bibliography{mal_ret}
% \begin{thebibliography}{1}

% \bibitem{kour2014real}
% George Kour and Raid Saabne.
% \newblock Real-time segmentation of on-line handwritten arabic script.
% \newblock In {\em Frontiers in Handwriting Recognition (ICFHR), 2014 14th
%   International Conference on}, pages 417--422. IEEE, 2014.

% \bibitem{kour2014fast}
% George Kour and Raid Saabne.
% \newblock Fast classification of handwritten on-line arabic characters.
% \newblock In {\em Soft Computing and Pattern Recognition (SoCPaR), 2014 6th
%   International Conference of}, pages 312--318. IEEE, 2014.

% \bibitem{hadash2018estimate}
% Guy Hadash, Einat Kermany, Boaz Carmeli, Ofer Lavi, George Kour, and Alon
%   Jacovi.
% \newblock Estimate and replace: A novel approach to integrating deep neural
%   networks with existing applications.
% \newblock {\em arXiv preprint arXiv:1804.09028}, 2018.

% \end{thebibliography}


\end{document}
