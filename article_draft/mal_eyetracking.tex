\documentclass{article}


\usepackage{arxiv}

\usepackage[utf8]{inputenc} % allow utf-8 input
\usepackage[T1]{fontenc}    % use 8-bit T1 fonts
\usepackage{hyperref}       % hyperlinks
\usepackage{url}            % simple URL typesetting
\usepackage{booktabs}       % professional-quality tables
\usepackage{amsfonts}       % blackboard math symbols
\usepackage{nicefrac}       % compact symbols for 1/2, etc.
\usepackage{microtype}      % microtypography
\usepackage{lipsum}

\title{Coarticulation and lexical access in Malayalam coronal stops}


\author{
  Indranil Dutta\thanks{The names of the authors appear alphabetically by last name. The study was conceptualized by all the authors together. The stimuli for the eye-tracking study was prepared by the second author. The eye-tracking experiment was conducted by the third author. The first, second, and fourth authors wrote different sections of the paper relevant to their contributions.} \\
  Speech and Language Processing Laboratory \\
  Department of Computational Linguistics\\
  The EFL University\\
  Hyderabad, TS 500007 \\
  \texttt{indranil@efluniversity.ac.in} \\
  %% examples of more authors
   \And
   Meghavarshini Krishnaswamy\\
  Speech and Language Processing Laboratory \\
  Department of Computational Linguistics\\
  The EFL University\\
  Hyderabad, TS 500007\\
  \texttt{meghavarshini@gmail.com } \\
  \And
  Ramesh Kumar Mishra \\
  Action Control and Cognition Lab \\
  Center for Neural and Cognitive Sciences (CNCS) \\
  University of Hyderabad\\
  Hyderabad, TS 500046\\
  \texttt{rkmishra@uohyd.ac.in} \\
  \And
  Seema Gorur Prasad \\
  Action Control and Cognition Lab \\
  Center for Neural and Cognitive Sciences (CNCS) \\
  University of Hyderabad\\
  Hyderabad, TS 500046\\
  \texttt{gp.seema@gmail.com} \\
  %% \AND
  %% Coauthor \\
  %% Affiliation \\
  %% Address \\
  %% \texttt{email} \\
  %% \And
  %% Coauthor \\
  %% Affiliation \\
  %% Address \\
  %% \texttt{email} \\
  %% \And
  %% Coauthor \\
  %% Affiliation \\
  %% Address \\
  %% \texttt{email} \\
}

\begin{document}
\maketitle

\begin{abstract}
This is where the abstract will come.
\end{abstract}


% keywords can be removed
\keywords{Coronal contrasts, lexical access, visual world, Malayalam}


\section{Introduction}
This is where the introduction will come. Assigned to ID.


\section{Materials and methods}
\label{sec:MandM}

This is where the materials and methods will come.

\subsection{Materials}
This is where the materials will come. Assigned to MK.

\subsection{Methods}

xx participants (xx female, mean age = x years, SD = ) took part in the study. All participants were students at University of Hyderabad and had acquired Malayalam as their L1. All participants reported normal or corrected-to-normal vision. 

Eye movement data was recorded using a desktop mounted Eyelink 1000 eyetracker (SR research, Ontario) with a sampling rate of 1000 Hz under binocular viewing.Participants rested their head on a chin rest for stable viewing. A 9-point calibration
was used for each participant. Stimuli was presented on a 19 inch LCD monitor with a refresh rate of 60 Hz placed at a distance of 60 cm from the participant.

Each trial began with a fixation cross for 1000 ms. A preview of the visual world display consisting of two printed Malayalam words was presented for 500 ms. Next, the Malayalam spoken word was presented and the visual world display was shown for 2500 ms longer. All the stimuli were presented in black against a grey background. 

There were four types of trials: 1) match trials where the spoken word contained matching coarticulatory information to one of the printed words. The other word in the display was an unrelated distractor. These trials were included to serve as a baseline to demonstrate that our paradigm can capture the biased looks towards spoken word referents. 2) mismatch trials with source word where the spoken word contained mismatched coarticulatory information with one of the printed words.The other word in the display was the source word that matched the coarticulatory information in the spoken word. 3) mismatch trials without source word where the spoken word included mismatched coarticulatory information. The display contained the printed version of the spoken word and another word that was phonologically similar to the spoken word (but didn't match the coarticulatory information). 4) filler trials where both the words in the display were unrelated to the spoken word. The filler trials were included to break anticipation and strategy.

Each participant was administered 10 match trials, 10 mismatch trials with source word, 6 mismatch trials without source word and 10 filler trials.  


\section{Results}
\label{sec:results}
This where the results will come. Assigned to ID, MK, and SGP. Add and cite references from malret.bib. For example: \cite{adank04,tabain1999,pandey2006}. If you are using SublimeText as your text editor along with LatexTools then enable `traditional' as the bibliography in the user settings file and it will automatically complete citations from the bib file.

\section{Discussion}
\label{sec:disc}
This is where the discussion section will come. Assigned to ID.

% The documentation for \verb+natbib+ may be found at
% \begin{center}
%   \url{http://mirrors.ctan.org/macros/latex/contrib/natbib/natnotes.pdf}
% \end{center}
% Of note is the command \verb+\citet+, which produces citations
% appropriate for use in inline text.  For example,
% \begin{verbatim}
%    \citet{hasselmo} investigated\dots{}
% \end{verbatim}
% produces
% \begin{quote}
%   Hasselmo, et al.\ (1995) investigated\dots
% \end{quote}

% \begin{center}
%   \url{https://www.ctan.org/pkg/booktabs}
% \end{center}


% \subsection{Figures}
% \lipsum[10] 
% See Figure \ref{fig:fig1}. Here is how you add footnotes. \footnote{Sample of the first footnote.}
% \lipsum[11] 

% \begin{figure}
%   \centering
%   \fbox{\rule[-.5cm]{4cm}{4cm} \rule[-.5cm]{4cm}{0cm}}
%   \caption{Sample figure caption.}
%   \label{fig:fig1}
% \end{figure}

% \subsection{Tables}
% \lipsum[12]
% See awesome Table~\ref{tab:table}.

% \begin{table}
%  \caption{Sample table title}
%   \centering
%   \begin{tabular}{lll}
%     \toprule
%     \multicolumn{2}{c}{Part}                   \\
%     \cmidrule(r){1-2}
%     Name     & Description     & Size ($\mu$m) \\
%     \midrule
%     Dendrite & Input terminal  & $\sim$100     \\
%     Axon     & Output terminal & $\sim$10      \\
%     Soma     & Cell body       & up to $10^6$  \\
%     \bottomrule
%   \end{tabular}
%   \label{tab:table}
% \end{table}



\bibliographystyle{unsrt}  
%\bibliography{references}  %%% Remove comment to use the external .bib file (using bibtex).
%%% and comment out the ``thebibliography'' section.


%%% Comment out this section when you \bibliography{mal_ret.bib} is enabled.
\bibliography{mal_ret}
% \begin{thebibliography}{1}

% \bibitem{kour2014real}
% George Kour and Raid Saabne.
% \newblock Real-time segmentation of on-line handwritten arabic script.
% \newblock In {\em Frontiers in Handwriting Recognition (ICFHR), 2014 14th
%   International Conference on}, pages 417--422. IEEE, 2014.

% \bibitem{kour2014fast}
% George Kour and Raid Saabne.
% \newblock Fast classification of handwritten on-line arabic characters.
% \newblock In {\em Soft Computing and Pattern Recognition (SoCPaR), 2014 6th
%   International Conference of}, pages 312--318. IEEE, 2014.

% \bibitem{hadash2018estimate}
% Guy Hadash, Einat Kermany, Boaz Carmeli, Ofer Lavi, George Kour, and Alon
%   Jacovi.
% \newblock Estimate and replace: A novel approach to integrating deep neural
%   networks with existing applications.
% \newblock {\em arXiv preprint arXiv:1804.09028}, 2018.

% \end{thebibliography}


\end{document}
